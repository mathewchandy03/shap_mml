%%%%%%%% ICML 2025 EXAMPLE LATEX SUBMISSION FILE %%%%%%%%%%%%%%%%%

\documentclass{article}

% Recommended, but optional, packages for figures and better typesetting:
\usepackage{microtype}
\usepackage{graphicx}
\usepackage{subfigure}
\usepackage{booktabs} % for professional tables

% hyperref makes hyperlinks in the resulting PDF.
% If your build breaks (sometimes temporarily if a hyperlink spans a page)
% please comment out the following usepackage line and replace
% \usepackage{icml2025} with \usepackage[nohyperref]{icml2025} above.
\usepackage{hyperref}


% Attempt to make hyperref and algorithmic work together better:
\newcommand{\theHalgorithm}{\arabic{algorithm}}

% Use the following line for the initial blind version submitted for review:
\usepackage{icml2025}

% If accepted, instead use the following line for the camera-ready submission:
% \usepackage[accepted]{icml2025}

% For theorems and such
\usepackage{amsmath}
\usepackage{amssymb}
\usepackage{mathtools}
\usepackage{amsthm}

% if you use cleveref..
\usepackage[capitalize,noabbrev]{cleveref}

%%%%%%%%%%%%%%%%%%%%%%%%%%%%%%%%
% THEOREMS
%%%%%%%%%%%%%%%%%%%%%%%%%%%%%%%%
\theoremstyle{plain}
\newtheorem{theorem}{Theorem}[section]
\newtheorem{proposition}[theorem]{Proposition}
\newtheorem{lemma}[theorem]{Lemma}
\newtheorem{corollary}[theorem]{Corollary}
\theoremstyle{definition}
\newtheorem{definition}[theorem]{Definition}
\newtheorem{assumption}[theorem]{Assumption}
\theoremstyle{remark}
\newtheorem{remark}[theorem]{Remark}

% Todonotes is useful during development; simply uncomment the next line
%    and comment out the line below the next line to turn off comments
%\usepackage[disable,textsize=tiny]{todonotes}
\usepackage[textsize=tiny]{todonotes}


% The \icmltitle you define below is probably too long as a header.
% Therefore, a short form for the running title is supplied here:
\icmltitlerunning{Supplementary Material for ``Conditional Conformal Intervals for Shapley Value
of Modalities in Machine Learning"}

\begin{document}

\onecolumn
\icmltitle{Supplementary Material for ``Conditional Conformal Intervals for Shapley Value
of Modalities in Machine Learning"}

% It is OKAY to include author information, even for blind
% submissions: the style file will automatically remove it for you
% unless you've provided the [accepted] option to the icml2025
% package.

% List of affiliations: The first argument should be a (short)
% identifier you will use later to specify author affiliations
% Academic affiliations should list Department, University, City, Region, Country
% Industry affiliations should list Company, City, Region, Country

% You can specify symbols, otherwise they are numbered in order.
% Ideally, you should not use this facility. Affiliations will be numbered
% in order of appearance and this is the preferred way.
\icmlsetsymbol{equal}{*}

\begin{icmlauthorlist}
\icmlauthor{Firstname1 Lastname1}{equal,yyy}
\icmlauthor{Firstname2 Lastname2}{equal,yyy,comp}
\icmlauthor{Firstname3 Lastname3}{comp}
\icmlauthor{Firstname4 Lastname4}{sch}
\icmlauthor{Firstname5 Lastname5}{yyy}
\icmlauthor{Firstname6 Lastname6}{sch,yyy,comp}
\icmlauthor{Firstname7 Lastname7}{comp}
%\icmlauthor{}{sch}
\icmlauthor{Firstname8 Lastname8}{sch}
\icmlauthor{Firstname8 Lastname8}{yyy,comp}
%\icmlauthor{}{sch}
%\icmlauthor{}{sch}
\end{icmlauthorlist}

\icmlaffiliation{yyy}{Department of XXX, University of YYY, Location, Country}
\icmlaffiliation{comp}{Company Name, Location, Country}
\icmlaffiliation{sch}{School of ZZZ, Institute of WWW, Location, Country}

\icmlcorrespondingauthor{Firstname1 Lastname1}{first1.last1@xxx.edu}
\icmlcorrespondingauthor{Firstname2 Lastname2}{first2.last2@www.uk}

% You may provide any keywords that you
% find helpful for describing your paper; these are used to populate
% the "keywords" metadata in the PDF but will not be shown in the document
\icmlkeywords{Machine Learning, ICML}

\vskip 0.3in


% this must go after the closing bracket ] following \twocolumn[ ...

% This command actually creates the footnote in the first column
% listing the affiliations and the copyright notice.
% The command takes one argument, which is text to display at the start of the footnote.
% The \icmlEqualContribution command is standard text for equal contribution.
% Remove it (just {}) if you do not need this facility.

%\printAffiliationsAndNotice{}  % leave blank if no need to mention equal contribution
\printAffiliationsAndNotice{\icmlEqualContribution} % otherwise use the standard text.



\section{Proofs}
\label{sec:proofs}

\begin{proof}[Proof of Lemma 3.1]
\label{proof:validity}
Because $\phi^{n+1}_j \in C_\text{split}(j) \iff \phi^{(\ell)}_j \leq \phi^{n+1}_j \leq \phi^{(u)}_j$,
$\mathbb P(\phi^{n+1}_j \in C_\text{split}(j)) = 
\mathbb P(\phi^{(\ell)}_j \leq \phi^{n+1}_j \leq \phi^{(u)}_j).$
By exchangeability of $\phi^{n+1}_j$ and $\phi^i_j, i \in \mathcal{I}_2$, which we re-index from $1$ to $m = n/2$,
the rank of $\phi^{n+1}_j$ among $\phi^1_j,...,\phi^m_j,\phi^{n+1}_j$ is
uniformly distributed over the set $\{1,...,m+1\}$. Therefore,
\begin{align*}
\mathbb P(\phi^{n+1}_j \in C_\text{split}(j)) &= 
\mathbb P(\phi^{n+1}_j \leq \phi^{(u)}_j) - \mathbb P(\phi^{n+1}_j < \phi^{(\ell)}_j) \\
&= \frac{u}{m+1} - \frac{\lfloor(m+1)(\alpha/2)\rfloor}{m+1}\\
&= \frac{\lceil (m+1)(1-\alpha/2)\rceil - \lfloor(m+1)(\alpha/2)\rfloor}{m+1} \\
&\geq 1- \alpha.    
\end{align*}

We define $D(j)$ such that 
$\phi^{n+1}_j \in D(j) \iff \phi^{n+1}_j < \phi^{(\ell)}_j$ or 
$\phi^{n+1}_j > \phi^{(u)}_j$. Therefore,
\begin{align*}
    \mathbb P(\phi^{n+1}_j \in D(j)) &= \mathbb P(\phi^{n+1}_j < \phi^{(\ell)}_j) + \mathbb P(\phi^{n+1}_j >\phi^{(u)}_j) \\
    &= \frac{\lfloor(m+1)(\alpha/2)\rfloor}{m+1} + \frac{m-u}{m+1} \\
    &= \frac{\lfloor(m+1)(\alpha/2)\rfloor +m-\lceil (m+1)(1-\alpha/2)\rceil}{m+1} \\
    &\leq \frac{(m+1)(\alpha/2)+(m+1)(\alpha/2)-1}{m+1}\\
    &=\alpha-\frac{1}{n/2+1} \\
    &= \alpha - \frac{2}{n+2}.
\end{align*}
And because $\mathbb P(\phi^{n+1}_j \in C_\text{split}(j)) = 1- \mathbb P(\phi^{n+1}_j \in D(j))$,
$$\mathbb P(\phi^{n+1}_j \in C_\text{split}(j)) = 1 - \alpha + \frac{2}{n+2}.$$

\end{proof}

\begin{proof}[Proof of Lemma 3.2]
We first examine the first order conditions of the convex optimization problem for some $\alpha$:
\begin{align*}
    \hat h_{\phi_j}^{1-\alpha} &= \underset{h \in \mathcal{H}}{\arg \min} \frac{1}{m+1} \sum_{i=m+1}^n \ell_\alpha(h(x^i), \phi^i_j) \ + \frac{1}{m+1}\ell_\alpha(h(x^{n+1}),\phi_j)+\mathcal{R}(h)\\
    &= \frac{1}{m+1}\sum_{i=m+1}^{n+1}\ell_\alpha(h(x^i),\phi_j^i)+\mathcal{R}(h).
\end{align*}
We have that for any fixed $h \in \mathcal{H}$,
$$0 \in \partial_\epsilon \Big (\frac{1}{m+1} \sum_{i=m+1}^{n+1} \ell_\alpha(\hat h^{1-\alpha}_{\phi_j}(x^i) + \epsilon f(x^i), \phi_j^i)+\mathcal{R}(\hat h^{1-\alpha}_{\phi_j} + \epsilon f) \Big ) \Big |_{\epsilon = 0}$$

Then the subgradients of the pinball loss term in the direction
of $f(.)$ are:
\begin{align*}
    \big \{\frac{1}{m+1}\big ( \sum_{\phi_j^i \neq \hat h^{1-\alpha}_{\phi_j }(x^i)} f(x^i)(\alpha - \mathbf 1\{ \phi^j_i > \hat h^\alpha_{\phi_j }\})\ + \sum_{\phi^i_j=\hat h^{1-\alpha}_{\phi_j }(x^i)}t^if(x^i)\big ) \big |t^i \in [\alpha-1,\alpha] \big \}
\end{align*}
Choose $t^{i*}_\alpha \in [\alpha-1,\alpha]$ such that they set
the subgradient to 0. Rearranging, we obtain
\begin{align*}
    \frac{1}{m+1}\sum_{i=m+1}^{n+1} f(x^i) (\alpha-\mathbf 1\{\phi^i_j>\hat h^{1-\alpha}_{\phi_j }(x^i)\}) = \frac{1}{m+1}\sum_{\phi^i_j = \hat h^{1-\alpha}_{\phi_j }(x^i)}(\alpha-t^{i*}_\alpha)f(x^i) -\frac{d}{d\epsilon}R(\hat h^{1-\alpha}_{\phi_j }+\epsilon f) \big |_{\epsilon=0}.
\end{align*}
We can relate the left hand side of the above equation to a
desired coverage guarantee as below:
\begin{equation}
\label{D1}
    \mathbb E[f(x^{n+1})(\mathbf 1\{\phi^i_j\in \hat C_j(x^{n+1})\}-(1-\alpha))] = \mathbb E[f(x^{n+1})(\alpha-\mathbf 1\{\phi^i_j >\hat h^{1-\alpha/2}_{\phi_j }(x^{n+1})\}-\mathbf 1\{\phi^i_j<\hat h^{\alpha/2}_{\phi_j}(x^{n+1})\})]
\end{equation}

Since both $\hat h^{1-\alpha/2}_{\phi_j }$ and $\hat h^{\alpha/2}_{\phi_j}$ are fit symmetrically, 
$\{(f(x^i), \hat h^{\alpha/2}_{\phi_j}, \hat h^{1-\alpha/2}_{\phi_j } (x^i), \phi^i_j)\}_{i=1}^{n+1}$ are exchangeable with each other. Thus,
\begin{multline*}
\label{D1}
    \mathbb E[f(x^{n+1})(\alpha-\mathbf 1\{\phi^i_j >\hat h^{1-\alpha/2}_{\phi_j }(x^{n+1})\}-\mathbf 1\{\phi^i_j<\hat h^{\alpha/2}_{\phi_j}(x^{n+1})\})] \\
    = \mathbb E[f(x^{n+1})(\alpha/2-\mathbf 1\{\phi^i_j >\hat h^{1-\alpha/2}_{\phi_j }(x^{n+1})\})\ 
    +f(x^{n+1})(\alpha/2 -\mathbf 1\{\phi^i_j<\hat h^{\alpha/2}_{\phi_j}(x^{n+1})\})] \\
    = \mathbb E[f(x^{n+1})(\alpha/2-\mathbf 1\{\phi^i_j >\hat h^{1-\alpha/2}_{\phi_j }(x^{n+1})\})-f(x^{n+1})(1-\alpha/2 -\mathbf 1\{\phi^i_j>\hat h^{\alpha/2}_{\phi_j}(x^{n+1})\})] \\
    = \mathbb E\Big [\frac{1}{m+1}\sum_{i=m+1}^{n+1}f(x^i)(\alpha/2-\mathbf 1\{\phi^i_j >\hat h^{1-\alpha/2}_{\phi_j }(x^i)\}) - \frac{1}{m+1}\sum_{i=m+1}^{n+1}f(x^i)(1-\alpha/2-\mathbf 1\{\phi^i_j >\hat h^{\alpha/2}_{\phi_j }(x^i)\})\Big ] \\
    = \mathbb E\Big [\frac{1}{m+1}\sum_{i=m+1}^{n+1}(\alpha/2 - t^{i*}_{\alpha/2})f(x^i)\mathbf 1\{\phi^i_j= \hat h^{1-\alpha/2}_{\phi_j }(x^i)\}\Big ]- \mathbb E \Big [\frac{d}{d\epsilon}R(\hat h^{1-\alpha/2}_{\phi_j } + \epsilon f) \Big |_{\epsilon=0} \Big] -\\
    \mathbb E\Big [\frac{1}{m+1}\sum_{i=m+1}^{n+1}(1 - \alpha/2 - t^{i*}_{1 - \alpha/2})f(x^i)\mathbf 1\{\phi^i_j= \hat h^{\alpha/2}_{\phi_j }(x^i)\}\Big ]+ \mathbb E \Big [\frac{d}{d\epsilon}R(\hat h^{\alpha/2}_{\phi_j } + \epsilon f) \Big |_{\epsilon=0} \Big].
\end{multline*}
Since $\alpha -t^{i*}_{\alpha} \in [0,1]$, for non-negative $f$, we have lower bound
$$\frac{1}{m+1}\sum_{i=m+1}^{n+1}(\alpha - t^{i*}_{\alpha})f(x^i)\mathbf 1\{\phi^i_j= \hat h^{1-\alpha}_{\phi_j }(x^i)\}\geq0.$$
Applying this to Equation~\ref{D1} gives us the first part
of Theorem 2:
\begin{multline*}
    \mathbb E [f(x^{n+1})(\mathbf 1\{\phi^i_j\in\hat C_j(x^{n+1})\}-(1-\alpha))] \geq - \mathbb E \Big [\frac{d}{d\epsilon}R(\hat h^{1-\alpha/2}_{\phi_j } + \epsilon f) \Big |_{\epsilon=0} \Big] + \mathbb E \Big [\frac{d}{d\epsilon}R(\hat h^{\alpha/2}_{\phi_j } + \epsilon f) \Big |_{\epsilon=0} \Big].
\end{multline*}

Relaxing the assumption of $f$'s non-negativity and using the exchangeability property, we also have upper bound
\begin{multline*}
\Big |\mathbb E \Big [ \frac{1}{m+1}\sum_{i=m+1}^{n+1}(\alpha/2 - t^{i*}_{\alpha/2})f(x^i)\mathbf 1\{\phi^i_j= \hat h^{1-\alpha/2}_{\phi_j }(x^i)\} \Big ] - \\
\mathbb E\Big [\frac{1}{m+1}\sum_{i=m+1}^{n+1}(1 - \alpha/2 - t^{i*}_{1 - \alpha/2})f(x^i)\mathbf 1\{\phi^i_j= \hat h^{\alpha/2}_{\phi_j }(x^i)\}\Big ] \Big | \\ 
\leq 
\Big |\mathbb E \Big [ \frac{1}{m+1}\sum_{i=m+1}^{n+1}(\alpha/2 - t^{i*}_{\alpha/2})f(x^i)\mathbf 1\{\phi^i_j= \hat h^{1-\alpha/2}_{\phi_j }(x^i)\} \Big ]\Big | + \\
\Big |\mathbb E\Big [\frac{1}{m+1}\sum_{i=m+1}^{n+1}(1 - \alpha/2 - t^{i*}_{1 - \alpha/2})f(x^i)\mathbf 1\{\phi^i_j= \hat h^{\alpha/2}_{\phi_j }(x^i)\}\Big ] \Big | \\
\leq \mathbb E\Big [\frac{1}{m+1} \sum_{i=m+1}^{n+1}|f(x^i)|\mathbf 1\{\phi^i_j= \hat h^{1-\alpha/2}_{\phi_j }(x^i)\} \Big ] \ + \\
\mathbb E \Big [\frac{1}{m+1} \sum_{i=m+1}^{n+1}|f(x^i)|\mathbf 1\{\phi^i_j= \hat h^{\alpha/2}_{\phi_j }(x^i)\}\Big ] \\
= \mathbb E[|f(x^i)| \mathbf 1\{\phi^i_j= \hat h^{1-\alpha/2}_{\phi_j }(x^i)\}] \ +
\mathbb E[|f(x^i)| \mathbf 1\{\phi^i_j= \hat h^{\alpha/2}_{\phi_j }(x^i)\}],
\end{multline*}
which gives us the second part of Theorem 2.
\end{proof}




\begin{lemma}
Assume that $(X^i, Y^i),..., (X^{n+1}, Y^{n+1}) \overset{i.i.d.}\sim P$ and that $K$ is uniformly bounded. Furthermore, for $j \in \{1,...,J\}$, suppose Assumption \ref{assumption:moment} holds and that the distribution of $\phi_j|X$ is continuous with a uniformly bounded density. Then
for any $f \in \mathcal H$ and $j \in \{1,...,J\}$,
$$\frac{|\epsilon_{int}|}{\mathbb E_P[|f(X)|]}\leq O\Big(\frac{d\log(n)}{\lambda n}\Big )\frac{\mathbb E[\max_{1\leq i \leq n+1}|f(X^i)|]}{\mathbb E_P[|f(X)|]}$$
\end{lemma}

\begin{proof}[Proof of Interpolation Error Bound]
The proof directly applies the results from \citet{gibbs2025conformal} and uses the stability of RKHS regression, but replacing the non-conformal score
with the Shapley value.

    
\end{proof}

\begin{assumption}[Bounded Utility]
\label{bounded_utility}
    Assume the utility function is bounded, i.e., $|\phi_j| \leq B$  almost surely.
\end{assumption}

It is known that Assumption~\ref{bounded_utility} holds for utility 
Shannon mutual information with $B = \min\{H(S),H(Y)\}$,
and for variance of conditional expectation
with $B = \text {Var}(X)$.

\begin{lemma}[RKHS Width Bound]
\label{rkhs_width_bound}
Let $\kappa = \sup_x \sqrt{K(x,x)}$. Let $C_\Omega(x) = ||\Omega(x)||_2$ be the norm
of the finite-dimensional features at test point $x$.
We also assume that Assumption~\ref{bounded_utility}. Then the width $W_j(x^{n+1})$ of conditional interval $\hat C_j(x^{n+1})$ with regularization parameter $\lambda$
satisfies:
$$W(x^{n+1}) \leq 2(\kappa\sqrt{\frac B \lambda} + C_\Omega(x)\sqrt\frac{B}{\lambda_2})$$
\end{lemma}

\begin{proof}[Proof of Lemma \ref{rkhs_width_bound}]
    Since $\hat h^{1-\alpha}_{\phi_j }$ minimizes
$$J(h_{\phi_j}^{1-\alpha})=\frac{1}{m+1} \sum_{i=m+1}^{n+1} \ell_\alpha(h^{1-\alpha}_{\phi_j}(x^i), \phi_j^i)+\lambda_1||h_K||^2_K + \lambda_2||\beta||^2_2,$$
$J(\hat h^{1-\alpha}_{\phi_j })$ must be lower
than the objective value of the zero function ($h_{\phi_j}^{1-\alpha} = 0$). Additionally, since
$\ell_\alpha(0, \phi_j) \leq \max(\alpha, 1-\alpha)|\phi_j|$, because we set both $h_k$ and $\beta$ to 0, and because we assume Assumptions~\ref{bounded_utility},
$$J(0)=\frac{1}{m+1} \sum_{i=m+1}^{n+1} \ell_\alpha(h^{1-\alpha}_{\phi_j}(x^i), \phi_j^i) + 0 \leq B.$$
Then,
$$\lambda_1||\hat h_K||^2_K \leq J(\hat h) \leq B,$$ or
$$||\hat h_K||_K \leq\sqrt{\frac{B}{\lambda_1}}.$$
Similarly,
$$\lambda_2||\hat \beta||^2_2 \leq J(\hat h) \leq B,$$ or
$$||\hat \beta||_2 \leq\sqrt{\frac{B}{\lambda_2}}.$$

We exploit the reproducing property of the RKHS,
which states that $h_K(x) = \langle h_K, K(x,\cdot)\rangle_K| \leq ||h||.$
Using the Cauchy-Schwarz inequality, we get:
$$|h_K(x)| = |\langle h_K, K(x,\cdot)\rangle_K| \leq 
||h_K||_K \cdot ||K(x, \cdot)||_K.$$
Again, from the reproducing property and Cauchy-Schwarz, $$||K(x,\cdot)||_K = \sqrt{\langle K(x,\cdot), K(x,\cdot)\rangle} = \sqrt{K(x,x)} \leq \kappa.$$
So 
\begin{align*}
    |\hat h(x)| &\leq \kappa||\hat h_K||_K + ||\hat \beta||_2 \cdot ||\Omega(x)||\\
    |\hat h(x)| &\leq \kappa\sqrt{\frac{B}{\lambda_1}} + C_\Omega(x) \sqrt{\frac{B}{\lambda_2}}
\end{align*}
$W(x) \leq |\hat h^U(x)| + |\hat h^L(x)|$
\end{proof}

In the case of classification, $B = \log(N_{classes})$
%In the case of regression, $B = Var(Y)$, so standard target to have
%unit variance










To show near-optimality of model selection for classification, we require two assumptions from \citet{he2024efficient}.
\begin{assumption}[$\epsilon$-Approximate Conditional Independence]
\label{assumption:conditional}
There exists a positive constant $\epsilon \geq 0$
such that, for all $S, S' \subseteq V$ with $S \cap S' = \emptyset$, we have $I(S;S'|Y) \leq \epsilon$
\end{assumption}

\begin{assumption}[$\epsilon$-Approximate Marginal Independence]
\label{assumption:marginal}
There exists a positive constant $\epsilon > 0$ such
that, for all $S, S' \subseteq V$ with $S \cap S'=\emptyset$, we have $I(S;S') \leq \epsilon$
    
\end{assumption}




To show near-optimality of model selection for regression, we consider a set $\Omega(V)=\{\Omega_j(X_j)\}_{j=1}^p$ of feature transformations such that two additional assumptions from \citet{he2024efficient} are satisfied.
\begin{assumption}[Linearity]
\label{assumption:linearity}
For some $\alpha$ and $\beta_j$ for $j \in \{1,...,p\}$,
    $$\mathbb E[Y | \Omega(V)]=\sum_{X_j \in V} \beta_j \Omega_j(X_j) + \alpha$$
\end{assumption}
\begin{assumption}[Multivariate Gaussian Marginal Distribution]
\label{assumption:multivariate}
    For some mean $\mu$ and covariance  matrix $\Sigma$,
    $$\Omega(V) \sim \mathcal N(\mu, \Sigma)$$
\end{assumption}

% \begin{theorem}[Selection Consistency for Classification]
% Define $$f_u(S) := \inf_{c \in \mathcal Y} \mathbb E[\ell(Y,c)]-\inf_{\mu_S \in \mathcal F} \mathbb E[\ell(Y, \mu_S(X_S))]$$
% Let $S(x)$ be the optimal set of size $q$. That is,
% $S(x) = \arg\max_{S:|S|\leq q} f_u(S|X=x)$
% Let $S^\alpha_q$ be the set of size $q$ selected
% by our algorithm for target 
% failure probability $\alpha \in (0,1)$.
% Under Assumptions~ \ref{assumption:moment}-\ref{assumption:marginal},
% the following inequality holds with probability
% at least $1 - \alpha - 2\lambda_1 \epsilon_1 - 2\lambda_2 \epsilon_2$:
% $$\text {val} (S^\alpha_q) \geq \text{val}(S^*)-\sum_{j\in S^*} 2(\kappa\sqrt{\frac B \lambda_1} + C_\Omega(x)\sqrt\frac{B}{\lambda_2}) - 4q \delta$$

    
% \end{theorem}


\begin{proof}[Proof of Theorem 3.7] 
Let $\hat h_j^L:=\hat h _{\phi_j}^{\frac{\alpha}{2q}}(x^{n+1})$
and let $\hat h_j^U := \hat h _{\phi_j}^{1-\frac{\alpha}{2q}}(x^{n+1})$.
    We define failure as the case where any
    individual modality in the selected set overestimates its value, or any modality
    in the optimal set underestimates its value.
    Let $f(\cdot) = f_K(\cdot)+\Omega(\cdot)^\top \beta$ indicate the
    weighting of interest and we deconstruct $\hat h_j^L = \hat h^L_{j,K}(\cdot)+\Omega(\cdot)^\top\hat \beta^{n+1}$.
    Applying a union bound and Lemma 2, we get:
    \begin{align*}
       \mathbb P(\underset{j \in \hat S^\alpha_q}\bigcup\phi_j^{n+1} < \hat h_j^L \underset{j \in S^*_q}\bigcup \phi_j^{n+1} > \hat h^U_j) &\leq
    \sum_{j\in \hat S^\alpha_q} \mathbb P(\phi_j^{n+1} < \hat h_j^L) + \sum_{j\in S^*_q} \mathbb P(\phi_j^{n+1} > \hat h_j^U) \\ 
    &\leq \sum_{j\in \hat S^\alpha_q}(\frac{\alpha}{2q})+
    \sum_{j\in S^*_q}(\frac{\alpha}{2q}) + 2\lambda_1 \epsilon_1+2\lambda_2 \epsilon_2 \\
    &\leq \frac{|\hat S^\alpha_q|+|S^*_q|}{2q}\alpha + 2\lambda_1\epsilon_1 + 2\lambda_2\epsilon_2\\
    &= \alpha + 2\lambda_1\epsilon_1 + 2\lambda_2\epsilon_2 +\sum_{j \in \hat S_q^\alpha}\frac{\epsilon_{int,j}}{E_P [f(X)]} + \sum_{j \in S^*_q}\frac{\epsilon_{int,j}}{E_P [f(X)]},
    \end{align*}
    where $$\epsilon_1=\sum_{j\in \hat S^\alpha_q} \frac{\mathbb E[\langle \hat h_{j, K}^{L}, f_K \rangle_K]}{\mathbb E_P [f(X)]}+
    \sum_{j\in S^*_q} \frac{\mathbb E[\langle \hat h_{j, K}^{L}, f_K \rangle_K]}{\mathbb E_P [f(X)]}$$
    and
    $$\epsilon_2 = \sum_{j\in \hat S^\alpha_q} \frac{\mathbb E[\langle \hat \beta^{n+1}, \beta \rangle]}{\mathbb E_P [f(X)]}+
    \sum_{j\in S^*_q} \frac{\mathbb E[\langle \hat \beta^{n+1}, \beta \rangle_K]}{\mathbb E_P [f(X)]},$$
    and modality interpolation errors $\epsilon_{int,j}$.

Following \citet{he2024efficient}, we choose
loss function $\ell(y, \hat y) := -\mathbf 1(y=1) \log \hat y - \mathbf 1(y=0) \log (1-\hat y)$. 
Because $\hat \mu_S$ minimizes $\mathbb E[\ell(Y, \mu_S(X_S)]$,
$f_u(S) = \mathbb E[\ell(Y,\hat \mu_\emptyset)] -\mathbb E[\ell(Y, \hat \mu_S(X_S))]$, with trivial predictor
$\mu_\emptyset$.

Proposition 3.2 from \citet{he2024efficient} demonstrates that the utility function is
monotonically non-decreasing, meaning that a larger set of modalities
will always have a utility function as high or higher than a smaller set.

We define the Shapley value of the utility function: 
$$\varphi_j = \sum_{S \subseteq [p] \setminus \{j\}} \frac{|S|!(p-|S|-1)!}{p!}[f_u(S \cup j) - f_u(S)]$$

Recall the Shapley value of the loss function for some point $i$:
$$\phi_j^i = \sum_{S \subseteq [p] \setminus \{j\}} \frac{|S|!(p-|S|-1)!}{p!}[\text{val}(y^i, \hat \mu_{S\cap j}(x^i_{S\cap j})) - \text{val}(y^i, \hat \mu_{S\cap j}(x^i_{S\cap j}))],$$
with $\text{val}(y^i, \hat \mu_{S\cap j}(x^i_{S\cap j}))=\ell(y^i, \hat \mu_\emptyset)-\ell(y^i,\hat \mu_S(x^i_S))$.
From linearity of expectation, it is easy to show
that the expected Shapley value of the loss function conditional on the training is
equivalent to the Shapley value of the utility function:
\begin{align*}
    \mathbb E[\phi_j^{n+1}|\hat \mu]&=\sum_{S\subseteq[p]\setminus\{j\}}\frac{|S|!(p-|S|-1)!}{p!} \{\mathbb E[\text{val}(Y, \hat \mu_{S\cup j}(X_{S\cup j}))] - \mathbb E[\text{val}(Y, \hat \mu_{S}(X_{S}))]\} \\
    &= \sum_{S\subseteq[p]\setminus\{j\}}\frac{|S|!(p-|S|-1)!}{p!} [f_u(S\cup j) - f_u(S)]\\
    &= \varphi_j
\end{align*}



Now that we have these pieces, we also apply Propositions 5.1, 5.2, and 5.3
from \citet{he2024efficient}, which under Assumptions~\ref{assumption:conditional} and \ref{assumption:marginal}, results in the following equation 
for some set $S$ of size at most$q$:
\begin{align*}
    \sum_{j\in S} I(X_j,Y) - q\delta \leq \sum_{j\in S} I(X_j,Y) - |S|\delta \leq\sum_{j\in S} \varphi_j &\leq \sum_{j\in S} I(X_j,Y) + |S|\delta \leq \sum_{j\in S} I(X_j,Y) + q\delta \\
    I(V_{S},Y) - 2q\delta \leq \sum_{j\in S} \varphi_j &\leq I(V_{S},Y) + 2q\delta \\
    f_u(S) - 2q\delta \leq \sum_{j\in S} \varphi_j &\leq f_u(S) + 2q\delta
    % &= \text{Var}(\mathbb E[Y|V_{S_q^\alpha}]) \\
    % &= \text{val}(S)
\end{align*}

Since the lower bound holds for all $j \in \hat S^\alpha_q$ and because 
we select a subset of size at most $q$ maximizing the sum of lower confidence bounds:
$$f_u(\hat S^\alpha_q) \geq \sum_{j\in \hat S^\alpha_q} \varphi_j - 2q\delta =\sum_{j\in \hat S^\alpha_q}\mathbb E[\phi_j^{n+1}]-2q\delta \geq \sum_{j\in \hat S^\alpha_q} \hat h_j^L - 2q\delta \geq\sum_{j\in S^*_q} \hat h_j^L - 2q\delta.$$
We also have that 
$$f_u(S^*_q) \leq \sum_{j\in \hat S^\alpha_q} \varphi_j +2q\delta =\sum_{j\in \hat S^\alpha_q}\mathbb E[\phi_j^{n+1}]+2q\delta\leq \sum_{j\in \hat S^\alpha_q} \hat h_j^U + 2q\delta \leq \sum_{j\in S^*_q} \hat h_j^U + 2q\delta.$$
Because $\hat h_j^L= \hat h_j^U - W_j$ for all $j \in \hat S^\alpha_q$
$$\mathbb \sum_{j\in \hat S^\alpha_q} (\hat h_j^U - W_j)=\mathbb \sum_{j\in \hat S^\alpha_q} \hat h_j^U - \mathbb \sum_{j\in \hat S^\alpha_q} W_j$$
Remember that the width $W_j$ of our interval is
bounded according to Lemma~\ref{rkhs_width_bound}.
Then,

\begin{align*}
    f_u(\hat S^\alpha_q) &\geq f_u(S^*_q) - \mathbb \sum_{j\in \hat S^\alpha_q} W_j - 4q\delta \\
    &= f_u(S^*_q) - \sum_{j\in \hat S^\alpha_q} 2(\kappa\sqrt{\frac B \lambda} + C_\Omega(x)\sqrt\frac{B}{\lambda_2}) - 4q\delta
\end{align*}

In the case of regression, we choose loss function
$\ell(y, \hat y) := (y-\hat y)^2$.
We also require a linear transformation $V'$ of $\Phi(V)$ from
Singular Value Decomposition (SVD):

    
\end{proof}

% \begin{theorem}[Selection Consistency for Regression]
% Let $S^*$ be the optimal set of size $q$.
% Let $S^\alpha_q$ be the set of size $q$ selected
% by our algorithm for target 
% failure probability $\alpha \in (0,1)$, and
% let $\ell(y, \hat y) = (y-\hat y)^2$
% Under Assumptions~ .........,
% the following inequality holds with probability
% at least $1 - \alpha - 2\lambda \epsilon_n$:
% $$\text {val} (S^\alpha_q) \geq \text{val}(S^*)-\sum_{j\in S^*} 2(\kappa\sqrt{\frac B \lambda} + C_\Omega(x)\sqrt\frac{B}{\lambda_2})$$


    
% \end{theorem}

\begin{proof}[Proof of Theorem 3.10] 
    The proof follows the proof of Theorem 3.7, but without a
    $\delta$ term. Here we replace
    \citet{he2024efficient}'s Proposition 3.2 with their Proposition 3.5,
    and Proposition 5.1, 5.2, and 5.3 with their
    Proposition 5.5, which allows strict additivity and monotonicity
    of Shapley values $\varphi$.
\end{proof}







\bibliography{main.bib}
\bibliographystyle{icml2025}





\end{document}


% This document was modified from the file originally made available by
% Pat Langley and Andrea Danyluk for ICML-2K. This version was created
% by Iain Murray in 2018, and modified by Alexandre Bouchard in
% 2019 and 2021 and by Csaba Szepesvari, Gang Niu and Sivan Sabato in 2022.
% Modified again in 2023 and 2024 by Sivan Sabato and Jonathan Scarlett.
% Previous contributors include Dan Roy, Lise Getoor and Tobias
% Scheffer, which was slightly modified from the 2010 version by
% Thorsten Joachims & Johannes Fuernkranz, slightly modified from the
% 2009 version by Kiri Wagstaff and Sam Roweis's 2008 version, which is
% slightly modified from Prasad Tadepalli's 2007 version which is a
% lightly changed version of the previous year's version by Andrew
% Moore, which was in turn edited from those of Kristian Kersting and
% Codrina Lauth. Alex Smola contributed to the algorithmic style files.
